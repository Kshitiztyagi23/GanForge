
\documentclass[12pt]{article}
\usepackage[a4paper, margin=1in]{geometry}
\usepackage{graphicx}
\usepackage{hyperref}
\usepackage{amsmath}
\title{Face Mask Detection Project Report}
\author{Beginner Deep Learning Project}
\date{\today}

\begin{document}

\maketitle

\section*{Introduction}
In this project, I built a face mask detection system using Convolutional Neural Networks (CNNs). The goal was to detect if a person is:
\begin{itemize}
    \item wearing a mask properly,
    \item not wearing a mask,
    \item wearing a mask incorrectly (e.g., below the nose).
\end{itemize}

\section*{Dataset}
I used the dataset from Kaggle: \href{https://www.kaggle.com/datasets/andrewmvd/face-mask-detection}{Face Mask Detection by Andrew Mvd}. It contains images labeled into three categories:
\begin{itemize}
    \item with\_mask
    \item without\_mask
    \item mask\_weared\_incorrect
\end{itemize}

\section*{Data Preprocessing}
\begin{itemize}
    \item Images were resized to a fixed size (128x128).
    \item Pixel values were normalized by dividing by 255.
    \item Data was split into training and validation sets.
    \item Used \textbf{ImageDataGenerator} from Keras for easy image loading and preprocessing.
\end{itemize}

\section*{Handling Imbalanced Data}
Since there were fewer examples of "mask\_weared\_incorrect", I calculated \textbf{class weights} to give more importance to that class during training.

\section*{Model Architecture}
A simple CNN was used:
\begin{itemize}
    \item 2 convolutional layers with ReLU activation
    \item Max pooling layers
    \item Flatten and Dense layers
    \item Output layer with 3 neurons and softmax activation (for the 3 classes)
\end{itemize}

\section*{Training}
The model was compiled using:
\begin{itemize}
    \item Loss: categorical\_crossentropy
    \item Optimizer: Adam
    \item Metrics: accuracy
\end{itemize}

I trained the model for 10 epochs using the computed \textbf{class weights} to help the model learn better for under-represented classes.

\section*{Testing and Visualization}
To test the model, I:
\begin{itemize}
    \item Detected faces using OpenCV’s Haar Cascade or MTCNN.
    \item Extracted the face region and passed it to the CNN.
    \item Predicted mask status and displayed the result with bounding boxes and labels.
\end{itemize}

\section*{Challenges}
\begin{itemize}
    \item False predictions when mask is worn partially.
    \item Some faces not detected properly due to overlapping or blur.
    \item MTCNN sometimes detects multiple overlapping boxes.
\end{itemize}

\section*{Future Improvements}
\begin{itemize}
    \item Use transfer learning with MobileNetV2 for better accuracy.
    \item Add Non-Maximum Suppression to remove duplicate detections.
    \item Train on more balanced data or use oversampling/augmentation.
\end{itemize}

\section*{Conclusion}
This was a great beginner-friendly project to understand CNNs, image classification, and handling real-world challenges like imbalanced data and face detection.

\end{document}
